\documentclass[12pt]{article}
\usepackage{fontenc}
\usepackage[french]{babel}
%\usepackage[utf8]{inputenc}
\usepackage{amsmath, amssymb}
\usepackage{geometry}
\usepackage{multicol}
\geometry{hmargin=2cm,vmargin=1.5cm}
\title{\textbf{MATH-F-112: Rappels}}
\author{Romain Grimau}
\date{}
\newcommand{\argsh}{\mathrm{argsh}} % argument sinus hyperbolique 
\newcommand{\argch}{\mathrm{argch}} % argument cosinus hyperbolique 
\newcommand{\ch}{\mathrm{ch}}
\newcommand{\sh}{\mathrm{sh}}

\begin{document}

\maketitle

\section*{S\'eance 1: logique, \'egalit\'e/in\'egalit\'e, r\'ecurrence}

implication : $\rightarrow$ \\
\indent double implication (si et seulement si): $\leftrightarrow$ \\
\indent et : $\land$ \\
\indent ou : $\lor$ \\
\indent n\'egation : $\lnot$ \\
\indent \textbf{S}uffisance implique \textbf{N}\'ecessaire: $S \rightarrow N$ \\
\indent contrapos\'ee de $A \rightarrow B$ = $\lnot A \rightarrow \lnot B$ et $(A\rightarrow B) \leftrightarrow (\lnot A \rightarrow \lnot B)$

\section*{S\'eance 2: nombres}

...

\section*{S\'eance 3: trigonom\'etrie}

\begin{multicols}{2}
$\cos(\alpha) = \cos(-\alpha)$\\
\indent$\sin(\pi - \alpha) = \sin(\alpha)$ \\
$\sin(-\alpha) = -\sin(\alpha)$ \\
$\cos(\pi - \alpha) = - \cos(\alpha)$
\end{multicols}
Dans triangle rectangle : $SOH - CAH - TOA - CAO$ \\ 
\indent Pythagore : $ hypot\acute enuse^{2} = adjacent^{2} + oppos\acute e^{2}$ \\
\indent Dans un triangle quelconque : $\frac{\sin(\hat{A})}{a} = \frac{sin(\hat{B})}{b} = \frac{sin(\hat{C})}{c}$ \\ 
\indent loi des cosinus : $c^{2} = a^{2}+b^{2} - 2ab\cos(\hat{c})$

\section*{S\'eance 4: fonctions}

$f : dom \rightarrow im$\\
\indent $dom f : \lbrace x|f(x) \exists \rbrace$, tous les $x$ pour lesquels il existe un $y$ \\ \newline
\indent Parit\'e :
$\left \lbrace \begin{tabular}{r c l }
	\(\forall x, f(x)\) & \(=\) & \(f(-x)\) PAIRE \\
	\(\forall x, f(x)\) & \(=\) & \(-f(x)\) IMPAIRE
\end{tabular}\right.$ \\ \newline
\indent Fonction p\'eriodique : si $\exists$ $p$ tel que $\forall x, f(x+p) = f(x)$ \\
\indent P\'eriodicit\'e: Si $f$ et $g$ sont p\'eriodiques, $f+g$ est p\'eriodique et une p\'eriode possible est le \indent$ppcm$ de la p\'eriode de $f$ et de la p\'eriode de $g$. 

\section*{S\'eance 5: combinatoire}

Choisir $\textbf{k}$ \'el\'ements parmis $\textbf{n}$, l'ordre n'a pas d'importance : \(C_n^k\) ou \(\binom{n}{k}\) \\ \newline
\indent Choisir $\textbf{k}$ \'el\'ements parmis $\textbf{n}$ sans r\'ep\'etition, l'ordre \`a de l'importance :
\begin{align*}
    n! & =  n\cdot(n-1)\cdot ... \cdot2\cdot1 \\
    A_n^k & =  \frac{n!}{(n-k)!}
\end{align*}
\indent Choisir $\textbf{k}$ \'el\'ements parmis $\textbf{n}$ avec r\'ep\'etition, l'ordre \`a de l'importance : \(n^k\)

\section*{S\'eance 6: g\'eom\'etrie analytique}

\begin{tabular}{lcl}
	 $(d)$ &$:$& $ax+by+cz+d = 0 \rightarrow \vec{v_{(d)}} = (-b, a)$\\
	 $(d')$&$:$& $a'x+b'y+c'z+d' = 0 \rightarrow \vec{v'_{(d')}} = (-b', a')$
\end{tabular} \\
\indent Si $(d // d') \Rightarrow \vec{v_{(d)}} = \vec{v'_{(d')}}$ \\ \newline
\indent$\left .
\begin{tabular}{lcl}
	 $(d)$ &$:$& $y = mx+p $\\
	 $(d')$&$:$& $y = m'x+p'$
\end{tabular}
\right \rbrace (d) // (d') \Rightarrow m = m'$ \\ \newline
\indent $(d) \perp (d') \Rightarrow m\cdot m' = -1$ \\

\'Equation du cercle de centre $(x_{c}, y_{c})$ et de rayon 1 : $(x-x_{c})^{2}+(y-y_{c})^{2} = 1^{2}$

\section*{S\'eance 7: g\'eom\'etrie analytique}

\'Equations du plan: \\
\indent \indent Pour $\vec{u}$ et $\vec{v}$, 2 vecteurs directeurs
\(\left \lbrace
   \begin{array}{rcl}
      x & = & x_0+\mu u_1+\lambda v_1 \\
      y & = & y_0+ \mu u_2+\lambda v_2 \\
      z & = & x_0+ \mu u_3+\lambda v_3
   \end{array}
\right. \) \\ \newline
\indent \indent Pour un vecteur normal $\vec{n} = (a, b, c)$: $ax+by+cz+d=0$ \\ \newline
\indent Produit scalaire de 2 vecteurs $\vec{u}, \vec{v}$:
$\vec{u}\cdot\vec{v} = \|\vec{u}\| \cdot \|\vec{v}\| \cdot \cos{\theta} = v_{1}u_{1} + v_{2}u_{2}+ ... + v_{n}u_{n}$ \\ \newline
\indent \'Equation d'une droite: \\
\indent \indent Pour $\vec{u}$, le vecteur directeur
\(\left \lbrace
	\begin{array}{rcl}
		x & = & x_0+tu_1 \\
		y & = & y_0+tu_2 \\
		z & = & z_0+tu_3
	\end{array} \right. \) $\rightarrow$ $\frac{x-x_0}{u_1} = \frac{y-y_0}{u_2} = \frac{z-z_0}{u_3}$ \\

\indent Produit vectoriel de $\vec{u}, \vec{v}$: $\vec{u} \times \vec{v} = (u_2v_3 - u_3v_2, u_3v_1 - u_1v_3, u_1v_2 - u_2v_1)$ \\

\indent Distance entre un point $(p_1, p_2, p_3)$ et une droite avec $\vec{n} = (a, b, c)$ : $\frac{ap_1+bp_2+cp_3+d}{\sqrt{a^2+b^2+c^2}}$ \\ \newline
\indent $Aire_{parall\acute ellogramme} = \|\vec{v_{1}} \times \vec{v_{2}} \|$
\indent $Aire_{triangle} = \frac{Aire_{parall\acute ellogramme}}{2}$

\section*{S\'eance 8: g\'eom\'etrie dans l'espace}
...

\section*{S\'eance 9: fonctions et \'equations trigonom\'etriques et logarithmes}

\begin{tabular}{r|ccccc}
	/ & $0$ & $\frac{\pi}{6}$ & $\frac{\pi}{4}$ & $\frac{\pi}{3}$ & $\frac{\pi}{2}$ \\
	\hline
	$\sin$ & $0$ & $\frac{1}{2}$ & $\frac{\sqrt{2}}{2}$ & $\frac{\sqrt{3}}{2}$ & $1$ \\
	$\cos$ & $1$ & $\frac{\sqrt{3}}{2}$ & $\frac{\sqrt{2}}{2}$ & $\frac{1}{2}$ & $0$\\
\end{tabular} \\ \newline
\indent$\log_{a} : ]0; \infty[ \rightarrow \mathbb{R}$ \\
\indent$x \rightarrow \log_{a}(x) = y$ o\`u $a^{y}=x$ \\ \newline
$\log_{a}(xy) = \log_{a}(x) + \log_{a}(y)$ \\
$\log_{a}(x^{n}) = n\log{a}(x)$ \\
$\log_{a}(a^{x}) = x = a^{\log_{a}(x)}$ \\
$\log_{a}(x) = \frac{\log_{b}(x)}{\log_{b}(a)}$ \\ 
$\log_{a}(b) = \frac{1}{\log_{a}(b)}$
\newline
$\sin{(a+b)} = \sin{a}\cos{b}+\cos{a}\sin{b}$ \\ 
$\cos{(a+b)} = \cos{a}\cos{b}-\sin{a}\sin{b}$

\section*{S\'eance 10: limites}

Si $\underset{x\to a}\lim f$ et $\underset{x\to a}\lim g$  $\exists$ $\Rightarrow$ 
\(
\begin{array}{lcl}
    \lim (f+g) & = & \lim f + \lim g \\
    \lim (f \cdot g) & = &  \lim f \cdot \lim g 
\end{array}\) \\ \newline
\indent Si $\underset{x\to a}\lim f$ et $\underset{x\to a}\lim g$  $\exists$ et $\underset{x\to a}\lim g \not= 0$ $\Rightarrow$ $\lim{\frac{f}{g}} = \frac{\lim{f}}{\lim{g}}$ \\ \newline
\indent Si $\underset{x\to a}\lim l = L$ et $\underset{t\to L}\lim g(t)$  $\exists$ $\Rightarrow$ $\underset{x\to a}\lim g(f(x)) = \underset{t\to L}\lim  g(t)$ \\
\section*{S\'eance 11: limites et asymptotes}
\noindent La droite $x = a$ est une asymptote verticale \`a $f$ si: $\underset{x\to a^{-}}\lim f(x) = \pm \infty$ ou $\underset{x\to a^{+}}\lim = \pm \infty$ \\ \newline

\noindent La droite $y = b$ est une asymptote horizontale \`a $f$ si: $\underset{x\to -\infty}\lim f(x) = b$ ou $\underset{x\to +\infty}\lim f(x) = b$ \\ \newline

\noindent La droite $y = mx + b$ est une asymptote oblique \`a $f$ si:\(\underset{x\to \pm \infty}\lim \frac{f(x)}{x} = m \not= 0 \) et \(\underset{x\to \pm \infty}\lim f(x)-m(x) = b \) \\ \newline
$f$ est asymptotiquement du m\^eme ordre que $g$ si : $\underset{x\to \infty}\lim |\frac{f(x)}{g(x)}| \in \mathbb{R}\backslash_{\{0\}}$
\newpage

\section*{S\'eance 12: d\'eriv\'ees}

Si $f$ est d\'erivable, alors $y = f'(a)(x-a)+f(a)$ est la droite tangente de $f$ au point $(a, f(a))$. \\
\indent $f'(a)$ est donc la pente de la tangente au point $(a, f(a)$. \\
\indent Si $f'(a) > 0 \Rightarrow f$ cro\^it. \\
\indent Si $f'(a) < 0 \Rightarrow f$ d\'ecro\^it.

\section*{S\'eance 13: d\'eriv\'ees}

$f'(a)$ est : 
\begin{itemize}\renewcommand{\labelitemi}{$\bullet$}
\item la pente de la tangente de $f$ au point $a$
\item la vitesse \`a laquelle $f$ cro\^it ou d\'ecro\^it en $a$
\end{itemize}
 
\indent $b$ est un point de max ou min si $f'(b) = 0$
\section*{S\'eance 14:Taylor}
Le polyn\^ome de Taylor de $f$ au point $a$ d'ordre $n$ est: \[T_{f, a, n}(x) = f(a) + f'(a)(x-a)+f''(a)\frac{(x-a)}{2}+...+f^{\underset{-}n}(a)\frac{(x-a)^{n}}{n!} = \sum_{k=0}^{n} f^{k}(a)\frac{(x-a)^{k}}{k!}\]
L'erreur commise $R_{n, a, f} = f^{(n+1)}(t)\frac{(x-a)^{n+1}}{(n+1)!}$ o\'u $t$ est entre $x$ et $a$ \\ \newline
D\'eveloppement de MacLaurin = Taylor en $0$.
\newpage

\section*{S\'eance 15: primitives}

$\int{f(x) dx} = F(x)$ \\
\indent tel que $F'(x) = f(x)$
\begin{multicols}{2}
\noindent1) $\int{n^{x} dx} = \frac{x^{n+1}}{n+1} + C$ \\
2) $\int{\frac{1}{x} dx} = \ln{x} + C$ \\
3) $\int{\frac{1}{1+x^{2}} dx} = \arctan{(x)} + C$ \\
4) $\int{\frac{1}{\sqrt{1-x^{2}}} dx} = \arcsin{(x)} + C$ \\
5) $\int{\frac{1}{\sqrt{1+x^{2}}} dx} = \argsh(x) + C$ \\
6) $\int{\frac{1}{\sqrt{x^{2}-1}} dx} = \argch(x) + C$ \\
7) $\int{\frac{1}{\cos^{2}(x)} dx} = \tan(x) + C$ \\
8) $\int{\frac{1}{\ch^{2}(x)} dx} = \tanh(x) + C$ \\
9) $\int{\frac{1}{\sin^{2}(x)} dx} = -\cot(x) + C$ \\
10) $\int{\cos(x) dx} = \sin(x) + C$ \\
11) $\int{\sin(x) dx} = -\cos(x) + C$ \\
12) $\int{\sh(x) dx} = \ch(x) + C$ \\
13) $\int{\ch(x) dx} = \sh(x) + C$ \\
14) $\int{e^{x} dx} = e^{x} + C$ \\
15) $\int{b^{x} dx} = \frac{b^{x}}{\ln(b)} + C$ \\
16) $\int{f(g(x))\cdot g'(x) dx} = F(g(x)) + C$ \\
17) $\int{\lambda f(x) dx} = \lambda \int{f(x) dx}$ \\
18) $\int{\frac{-1}{\sqrt{1-x^{2}}} dx} = \arccos(x) + C$ \\
19) $\mathrm{cosec}(x) = \frac{1}{\sin(x)}$ \\
20) $\int{k dx} = kx + C$
\end{multicols}

\section*{S\'eance 16: int\'egration par substitution}

$\int{f(g(x))\cdot g'(x) dx} = F'(g(x)) + C$ \\
\newline
L'id\'ee: on pose 
\(\begin{array}{ccc} 
t &=& g(x) \\ 
\frac{dt}{dx} &=& g'(x) 
\end{array} \) $\Rightarrow \int{f(g(x)) \cdot g'(x) dx} = \int{f(t) dt} = F(t) + C = F(g(x)) + C$ \\

\section*{S\'eance 17: int\'egration par partie}

$(fg)' = f'g + g'f \Rightarrow \int{f'g} = fg - \int{g'f}$ 

\section*{S\'eance 18: fractions simples et int\'egrales d\'efinies}

...

\section*{S\'eance 19: int\'egrales d\'efinies}

Si on int\`egre jusqu'en $\pm \infty$ ex: $\int_{0}^{\infty}{e^{-x} dx}$ \\

\begin{tabular}{ r c l }
  \( \Rightarrow \int_{0}{\infty}{e^{-x} dx} \)& \(=\) & \(\underset{u\to \infty}\lim \int_{0}^{u}{e^{-x} dx}\) \\
   & \(=\) & \(\underset{u\to \infty}\lim \lbrack e^{-x}\rbrack_{0}^{u}\) \\
   & \(=\) & \(\underset{u\to \infty}\lim{-e^{-u} + e^{-0}} = 1\)
\end{tabular} \\ \newline

Si on int\`egre une fonction non-born\'ee ex: $\int_{0}^{1}{\frac{1}{x} dx}$ \\
\begin{tabular}{r c l }
	\( \Rightarrow \int_{0}^{1}{\frac{1}{x} dx}\) & \( = \) & \(\underset{u\to 0}\lim{\int_{u}^{1}{\frac{1}{x} dx}}\) \\
	& \( = \) & \(\underset{u\to 0}\lim{\lbrack \ln(|x|) \rbrack_{u}^{0}}\) \\
	& \( = \) & \(\underset{u\to 0}\lim{\ln(1) - ln(u)} = -(-\infty) = \infty\)
\end{tabular}

\section*{S\'eance 20: courbes}

Si $\gamma(t)$ est la courbe position: \\
\begin{tabular}{r c c l }
	\(\Rightarrow\) & \(\vec{v}\) & \( = \) & \(\gamma(t)'\) est le vecteur vitesse (ou vecteur tangent)\\
	& \(\vec{a}\) & \( = \) & \(\vec{v}' = \gamma(t)''\) est le vecteur acc\'el\'eration (ou vecteur normal
\end{tabular} \\ \newline
norme : \(\|\vec{v}\| = \sqrt{v_{1}^{2} \cdot v_{2}^{2} \cdot ... \cdot v_{n}^{2} }\)

\section*{ S\'eance 21:matrices et syt\`emes}

 Inverse d'une matrice: $\begin{pmatrix}
a & b \\
c & d
\end{pmatrix}^{-1} = \frac{1}{det A} \cdot \begin{pmatrix} d & -b \\ -c & a \end{pmatrix}$

M\'ethode de Gauss : 
\(\left\lbrack\ \  A^{}\ \ \  
 \begin{array}{|cc}
  1&0\\
  0&1
 \end{array}\right\rbrack
\Rightarrow  \left\lbrack\
\begin{array}{cc|}
  1&0\\
  0&1
\end{array}\ \  A^{-1} \ \ \right\rbrack\) \\
\newline
Op\'erations sur un syst\`eme lin\'eaire homog\`ene:
\begin{itemize}
\item $L_{i} \Leftrightarrow \lambda L_{i}, \lambda \not= 0$ et $\lambda \in \mathbb{R}$
\item $L_{i} \Leftrightarrow L_{i} - \mu L_{j}, \mu \in \mathbb{R}$ 
\item $L_{i}= L_{j}, L_{j} = L_{i}$ (permutation)
\end{itemize}
\section*{Fin premier quadrimestre}
\paragraph{Pour des notes plus compl\`etes, voir la synth\`ese de Marie ;)}
\paragraph{Pour les notes les plus compl\`etes possible, voir le cours !}
\end{document}
