\documentclass[12pt]{article}
\usepackage{fontenc}
\usepackage[francais]{babel}
%\usepackage[utf8]{inputenc}
\usepackage{amsmath, amssymb}
\usepackage{geometry}
\geometry{hmargin=2cm,vmargin=1.5cm}
\title{\textbf{MATH-F-112: Rappels}}
\author{Romain Grimau}
\date{}
\begin{document}

\maketitle

\section*{S\'eance 1: logique, \'egalit\'e/in\'egalit\'e, r\'ecurrence}
...
\section*{S\'eance 2: nombres}
...
\section*{S\'eance 3:trigonom\'etrie}
...
\section*{S\'eance 4:fonctions}
...
\section*{S\'eance 5: combinatoire}
Choisir $\textbf{k}$ \'el\'ements parmis $\textbf{n}$, l'ordre n'a pas d'importance : \(C_n^k\) ou \(\binom{n}{k}\) \\ \newline
\indent Choisir $\textbf{k}$ \'el\'ements parmis $\textbf{n}$ sans r\'ep\'etition, l'ordre \`a de l'importance :
\begin{align*}
    n! & =  n\cdot(n-1)\cdot ... \cdot2\cdot1 \\
    A_n^k & =  \frac{n!}{(n-k)!}
\end{align*}
\indent Choisir $\textbf{k}$ \'el\'ements parmis $\textbf{n}$ avec r\'ep\'etition, l'ordre \`a de l'importance : \(n^k\)
\section*{S\'eance 6:}
\section*{S\'eance 7:g\'eom\'etrie analytique}
\'Equations du plan: \\
\indent \indent Pour $\vec{u}$ et $\vec{v}$, 2 vecteurs directeurs
\(\left \lbrace
   \begin{array}{rcl}
      x & = & x_0+\mu u_1+\lambda v_1 \\
      y & = & y_0+ \mu u_2+\lambda v_2 \\
      z & = & x_0+ \mu u_3+\lambda v_3
   \end{array}
\right. \) \\ \newline
\indent \indent Pour un vecteur normal $\vec{n} = (a, b, c)$: $ax+by+cz+d=0$ \\ \newline
\indent Produit scalaire de 2 vecteurs $\vec{u}, \vec{v}$:
$\vec{u}\cdot\vec{v} = \|\vec{u}\| \cdot \|\vec{v}\| \cdot \cos{\theta}$ \\ \newline
\indent \'Equation d'une droite: \\
\indent \indent Pour $\vec{u}$, le vecteur directeur
\(\left \lbrace
	\begin{array}{rcl}
		x & = & x_0+tu_1 \\
		y & = & y_0+tu_2 \\
		z & = & z_0+tu_3
	\end{array} \right. \) $\rightarrow$ $\frac{x-x_0}{u_1} = \frac{y-y_0}{u_2} = \frac{z-z_0}{u_3}$ \\

\indent Produit vectoriel de $\vec{u}, \vec{v}$: $\vec{u} \times \vec{v} = (u_2v_3 - u_3v_2, u_3v_1 - u_1v_3, u_1v_2 - u_2v_1)$ \\

\indent Distance entre un point $(p_1, p_2, p_3)$ et une droite avec $\vec{n} = (a, b, c)$ : $\frac{ap_1+bp_2+cp_3+d}{\sqrt{a^2+b^2+c^2}}$\\
\section*{S\'eance 8:}
\section*{S\'eance 9:fonctions et \'equations trigonom\'etriques et logarithmes}
\section*{S\'eance 10:limites}
Si $\underset{x\to a}\lim f$ et $\underset{x\to a}\lim g$  $\exists$ $\Rightarrow$ 
\(
\begin{array}{lcl}
    \lim (f+g) & = & \lim f + \lim g \\
    \lim (f \cdot g) & = &  \lim f \cdot \lim g 
\end{array}\) \\ \newline
\indent Si $\underset{x\to a}\lim f$ et $\underset{x\to a}\lim g$  $\exists$ et $\underset{x\to a}\lim g \not= 0$ $\Rightarrow$ $\lim{\frac{f}{g}} = \frac{\lim{f}}{\lim{g}}$ \\ \newline
\indent Si $\underset{x\to a}\lim l = L$ et $\underset{t\to L}\lim g(t)$  $\exists$ $\Rightarrow$ $\underset{x\to a}\lim g(f(x)) = \underset{t\to L}\lim  g(t)$ \\
\section*{S\'eance 11: limites et asymptotes}
La droite $x = a$ est une asymptote verticale \`a $f$ si: $\underset{x\to a^{-}}\lim f(x) = \pm \infty$ ou $\underset{x\to a^{+}}\lim = \pm \infty$ \\ \newline

La droite $y = b$ est une asymptote horizontale \`a $f$ si: $\underset{x\to -\infty}\lim f(x) = b$ ou $\underset{x\to +\infty}\lim f(x) = b$ \\ \newline

La droite $y = mx + b$ est une asymptote oblique \`a $f$ si:\(\underset{x\to \pm \infty}\lim \frac{f(x)}{x} = m \not= 0 \) et \(\underset{x\to \pm \infty}\lim f(x)-m(x) = b \) \\ \newline
$f$ est asymptotiquement du m\^eme ordre que $g$ si : $\underset{x\to \infty}\lim |\frac{f(x)}{g(x)}| \in \mathbb{R}\backslash_{\{0\}}$
\section*{S\'eance 12:dérivées}

\section*{S\'eance 13:dérivées}
...
\section*{S\'eance 14:Taylor}
...
\section*{S\'eance 15:primitives}
...
\section*{S\'eance 16:primitives}
...
\section*{S\'eance 17:primitives}
...
\section*{S\'eance 18:fractions simples et intégrales définies}
...
\section*{S\'eance 19:int\'egrales d\'efinies}
...
\section*{S\'eance 20:courbes}
...
\section*{S\'eance 21:matrices et sytm\`emes}
...


\end{document}
